% Use only LaTeX2e, calling the article.cls class and 12-point type.

\documentclass[12pt]{article}

\usepackage{scicite}
\usepackage{times}
\usepackage{graphicx} 
\usepackage{amsmath}


\topmargin 0.0cm
\oddsidemargin 0.2cm
\textwidth 16cm 
\textheight 21cm
\footskip 1.0cm


%The next command sets up an environment for the abstract to your paper.

\newenvironment{sciabstract}{%
\begin{quote} 
FARAD is an attempt at providing a new novel way of commoditising Intellectual Property Rights related to a set of technologies related to metal oxide based ultra-capacitors development and production. The approach taken here is by introducing forward contracts based on the manufacturing application of the technology, and how these forward contracts are structured as the underlying assets of a digital commodity. The commoditisation program entails identification of the economic appropriation rights for the technologies and how it could be coded onto Ethereum Blockchain Smart Contracts, which would then create a tokenise digital asset commodity called FARAD. This digital asset is named as a ``Cryptoken", signifying the cryptographic element of the Ethereum Blockchain and the tokenising of the digital asset. The paper sets out on how this could be achieved. 
\bf}{\end{quote}}


\renewcommand\refname{References and Notes}

\newcounter{lastnote}
\newenvironment{scilastnote}{%
\setcounter{lastnote}{\value{enumiv}}%
\addtocounter{lastnote}{+1}%
\begin{list}%
{\arabic{lastnote}.}
{\setlength{\leftmargin}{.22in}}
{\setlength{\labelsep}{.5em}}}
{\end{list}}

\title{FARAD: \\ Commoditising Forward Purchase Contracts \\in Ultra-capacitor Intellectual Property Rights\\ on Ethereum Blockchain} 

\author
{Wan M. Hasni$^{1}$
     and You Hua Nong.$^{2}$\\
\\
\normalsize{$^{1}$Chairman and CEO, Virtue Fintech FZ LLC.}\\
\normalsize{$^{2}$Chief Executive Officer, HK Aerospace Beidou New Energy Technology Co. Ltd.}\\
\\
\textsubscript{$^\ast$The authors benefited from comments and inputs from Andras Kristof.}
}


\date{7\textsuperscript{th} July 2017}



%%%%%%%%%%%%%%%%% END OF PREAMBLE %%%%%%%%%%%%%%%%



\begin{document} 

% Double-space the manuscript.

\baselineskip24pt

% Make the title.

\maketitle 
\graphicspath{ {Images/} }

\begin{sciabstract}
 
 
\end{sciabstract}

\section*{Introduction}

The economics of welfare and the allocation of resources for invention had been in the mind of economists such as Arrow (1962) for a very long time, and until today the issues remain unresolved and continue to plague innovators, industries, and governments alike. At the heart of the problem is how does resources (i.e. investments), be allocated to support the innovators to continue their production of knowledge (which we would define here as the Intellectual Property Rights or ``IPR"),\footnote{Intellectual property rights refers to the general term for the assignment of property rights through patents, copyrights, and trademarks. These property rights allow the holder to exercise a monopoly on the use of the item for a specified period.} so that society could progress. Too little of resources allocated would mean that society is hindered by the lack of technological progress; and at the same time, the high level of uncertainties facing any investments into IPR requires higher risk premiums. This leads to the pricing of such risks be indeterminate and the market for IPR be incomplete (as described by Arrow and Debreu (1954)). The incompleteness here emanates from both sides: the innovators (or producers of IPRs) and the investors (or resource providers). In economic of exchange environment, this is termed as the no-trade Nash equilibrium.    

Arrow (1962) suggests that we could model solutions for this problem by creating a market for ``commodity options", where information is a commodity, and as a commodity, it could possibly be traded. The main object of such commodity is that its owner could appropriate economic benefits by having sufficient legal measures in order for some form of monopoly power be exerted. Once this is defined, the risk measures could then be determined and would have a better chance of being priced efficiently.

The first part of the solution involves the protection of the IPR - where in today's world, the rudimentary form of information is in bits and bytes. The way to create ``protected commodity" is by creating a ``box", whereby such bits and bytes could be stored and disseminated under the ``public eye" and yet being protected from any theft. This box is what could be termed as data entry, storage, and publication in a publicly distributed ledger and database. Under such environment, the \textit{private} information such as an IPR, could be \textit{published publicly} with all the necessary identifiers, and yet, only people with \textit{permission} could see what is inside the ``box" without any compromise of information thievery. As an example, a patent could be submitted (by publishing publicly), and most contents of the patent details are hidden from the public (such as proprietary processes, etc.), save what is absolutely necessary to be disclosed (such as major economic claims). The second part of the solution (assuming that the box is already secured), is how to secure the economic appropriations from such ``box". If those economic claims could be defined and determined precisely, it would allow any economic investors to have sufficient information for them to make judgments on the risks related to such claims. Furthermore, if such process is transparent with a high level of accountability, and could be ratified within the market, then the risks associated with the ``box" and its claims could be ascertained definitively.

The blockchain technology so far is the most viable answer to the problem posed above - with its potential to deal with the essential basic building blocks of information (i.e. data), by the allowance of the creation of virtual ledger to be recorded and distributed with the principles of what is termed as \emph{``trust-less"} or \emph{``intermediary-less"} methods of transactions, as its core functionalities. It is a revolution for the fact that, these are the issues plaguing the whole financial industry - which blockchain could potentially solve.  

Bitcoin, as the main proof of concept and proof of technology for blockchain, had been operating successfully for more than eight years since its first creation in 2009.\footnote{Please refer to Nakamoto (2008).} It is the main reference point for the applications of blockchain technology. However, what had been less understood, is the economic potentials of the backbone of blockchain - that is the Distributed Ledger and Database Technology (DLDT). DLDT, while novel in its structure and forms, still hasn't realised its full economic potential. With the rise of many crypto-currencies, such as Bitcoin, Ethereum and alike, it was argued by its proponents that it could enter the market as a clear alternative for the furtherance of international exchange economy by being more efficient (lesser costs of transactions and less time consuming), information-ally more efficient (by having a fully decentralised and \emph{trust-less} system), whereby economic agents could operate in a \emph{border-less world} (hence overcoming barriers and constraints), and finally the boundaries of optimality and fairer economic allocations could be achieved.\footnote{For detail explanations about cryptocurrencies and Bitcoins, please see Vigna and Casey (2016).} 

Bitcoins has its own shares of controversies, and being the first one introduced have some inherent draw backs. However, many of these issues are addressed in the newer blockchain systems that had been introduced, such as the Ethereum (blockchain) Networks.\footnote{Please refer to the Ethereum Whitepaper, Buterin (2014).} One of the most significant approaches of Ethereum is what is termed as ``Smart Contracts" which makes it be more suited to customised  applications.\footnote{Ethereum Network is ``��a decentralised platform that runs smart contracts: applications that run exactly as programmed without any possibility of downtime, censorship, fraud or third party interference. Ethereum'�s protocol is built to allow flexibility and increase functionality to provide the ability to program many different types of smart contracts within the Ethereum system. Ethereum is written in Turing complete language". For reference, please see Harm, et. all (2017).} We will use the Ethereum platform as the base of our proposed solution.

In order to demonstrate the idea of commoditising Intellectual Property Rights as proposed, the paper will explain how Ethereum blockchain could be used as the backbone for the commoditisation process, and the result of such exercise would produce a crypto-token (or cryptocurrency) which could be traded and exchanged within the market place. If this idea could be implemented, it would open up a whole plethora of financing and investments in technologies, which would be a major push for many new technologies where this new method could be the choice of fundraising activities.  

The paper proceeds as follows: The \textbf{first section} will lay down the principles of economics of exchange and how could  \emph{``trust-less"} system of economic exchanges be performed and how the blockchain Distributed Ledger and Database Technology be applied theoretically to a system of economic exchanges, by deploying ``Smart Contracts", which will lead to how a crypto-token be created out of such process; the \textbf{second section} will describe a specific IPR, which is the deployment of technologies for the manufacturing of metal oxides based ultra-capacitors be commoditised within the definition of the crypto-token. The crypto-token is named as FARAD; the \textbf{third section} will explain how FARAD is generated using the Ethereum Network blockchain environment, and how further improvements for the FARAD ecosystem could be undertaken, and finally the \textbf{fourth section} concludes.


\section{Role of blockchain in economic exchanges} %Section 1

The main principle of economic of exchange is whenever parties could agree on the value of the items to be exchanged, be it goods, commodities, services, or monies - the exchange would occur, otherwise, the exchange failed to take place (or no-trade would happen). In order to describe this, the main descriptions of the exchange economy and the main assumptions to for an economic exchange model need to be defined.\footnote{Any forms of market is a form of economic exchange. A bazaar as a market place is an exchange as much as the stock markets. We are using a simplified version of economic exchange model by generalising some of the assumptions. For full understanding of the markets and rigorous treatments, readers are referred to MacMillan (2002) and Kreps (1990).}

\subsection{Descriptions of economic exchange}

An economic exchange could happen if two party, Alice, $A$ who owns some Commodity $x$ and Bob, $B$ who owns some Currency $y$, agree on exchanging between them, whereby Alice takes some units of the Currency $y$ in exchange of Bob taking some units of Commodity $x$. Negotiations happens at time $t_{0}$, and transaction took place at time $t_{1}$. This is only possible If and Only If, Alice and Bob agree on the values of those Commodity and Currency to be the same at $t_{0}$.

This can be represented by: 

\begin{equation} \label{eq:1}
    F_{A,t_{0}} (y_{t_{1}}) = G_{B,t_{0}} (x_{t_{1}})
\end{equation}

Where $A$ is Alice, and $B$ is Bob, and $F$ and $G$ are \emph{value functions} at time $t_{0}$, of Alice on $y$, and Bob on $x$, respectively. If the above equation fails to match, no exchange will occur, or no trade happened. The no-trade theorem\footnote{For detail exposition of ``Nash Equilibrium of No-Trade", please refer to standard economic text books on game theory such as Gintis (2000).} could be described by first expanding Equation 1 as follows:

\begin{equation}
    F_{A,t_{0}} (y_{t_{1}} + \delta_{y,t_{1}} )=G_{B,t_{0}} (x_{t_{1}} + \delta_{x,t_{1}})
\end{equation}

Where $\delta_{y,t_{1}}$ and $\delta_{x,t_{1}}$  are \emph{the differences or variations in expectation}, which represents the differences in the quality of the Commodity $x$ delivered, and the premiums or discounts in the amount of payments to be received in Currency $y$, at time $t_{1}$. In another word, there are uncertainties involved in the transaction in regard to the value of Commodity $x$ and amount of Currency $y$.\footnote{Note that Bob may pay in his local currency, which is different from what Alice received, which is in her local currency.} This could be viewed from the classic Nash's equilibrium solution to the Prisoner's Dilemma setting.\footnote{The term Prisoner's Dilemma is widely used and explained in most of the economic textbooks as part of economic game theoretical discussions. As a reference, please see Kreps(1990).}

What are the sources of variations which lead to failed trade? There are a plethora of things that could possibly happen. In the studies of economics of exchange, the possible candidates are: the costs of transactions, informational asymmetries or incomplete information, disjointed or fragmented market problems, economic externalities such as taxations, and others.\footnote{Detail discussions on the subject could be found in most classical economic texts, such as Akerlof (1970), Spence (1973), Grossman and Stiglitz (1980), and Kreps (1990).}

\subsection{Trust-less system of economic exchanges}

Now let us turn to another major issue in any economic exchanges namely what could be termed as \textit{trust-less} system of economic exchanges. It is well known that one of the central problems in economic exchanges is about ``Payments" versus ``Delivery" (``PVD") or sometimes also termed as ``Post Trade Settlements" (``POTS"). The standard solution for this problem is for the market to operate through ``market intermediaries" which functions as the central party playing the role of intermediation, as the custodian and clearing agents for both sides of the exchanges. For this to happen, all parties must ``trust" the intermediary to play its role. Banks, stock exchanges, fund managers - are among the notable examples of these ``trusted" intermediaries. Unfortunately, the intermediation process itself is mired in the question of ``trust" as well as ``agency problems", with which a set of layers and costs involved (such as transaction costs, monitoring costs, etc), opacity of the process (such as informational asymmetries), market fragmentation and disjointedness, and economic externalities.

The question is, could this be reduced or eliminated? This is what a system of ``trust-less" economic�� exchange could offer. Before proceeding further, it is important to lay down the basic principles of ``trust-less" economic� exchanges, which are necessary (but not necessarily sufficient) requirements for it to be possible.

\begin{enumerate}

\item \underline{Intermediary-less transactions} - there is no need of any central party or ``trusted" intermediaries to be a middle party to the transactions. The transacting parties are relying directly on each other through a system whereby ``trusting" someone is not relevant; but rather the ``trust" is being relied upon the system itself and the system is free from ``human interference". 

\item \underline{Ir-reversibility and immutability of transactions} - once an exchange is agreed, as per agreed value and terms, it is no longer changeable or mutable; no parties could unilaterally change or cancel unless pre-agreed arrangements had been made. Despite any arrangements made, it must be made and agreed within the public domain rather than on personalised domain, which makes such transactions to be immutable.

\item    \underline{Automatic execution of contracts} - once a contract is consummated, it will be executed without any further reference to any parties. The execution is on ``automated" track, whereby both parties have delegated the execution to a system whereby there is no need of exercising further efforts beyond the first exercise of contracting. The contracts should then be fully specified which leaves no room for discrepancies in terms of its executions.

\item \underline{Transparency, responsibility, and accountability} - the execution of the trade are within the public environment, where all items are fully transparent to everyone. Furthermore, the responsibilities are clearly defined and every party is fully responsible for their promises, and be accountable for it as well.

\end{enumerate}

Could a \emph{trust-less} system of economic exchanges with the above principles be conducted in practice? This is best explained through an example (following from the example of Alice and Bob).

\begin{quote}
Alice, who owns a commodity, wishes to sell the commodity on a forward sale contract to Bob. Bob agreed to pay for the commodity now, at a fixed price of say, $y_{1}$. In this case $y_{1}$ is assumed to be delivered to Alice on the spot, and Alice could accept $y_{1}$ as its the choice of payment currency (i.e. no foreign exchange risks taken by both sides). The forward sale contract specifies that at a certain fixed time in the future, Alice shall deliver the commodity into a pre-determined ``box", and when the time period matures (i.e. the delivery date), the commodity is delivered automatically into such ``box" and Alice has no influence to change such process (i.e. it is an automated process). This process had bet set and pre-determined even before the spot payments are made by Bob (and hence Bob is secured by such process, rather than dependence on Alice's performance based on her promises).

Furthermore, all the contracts and promises are done between Alice and Bob in a ``public domain", which would then have it recorded and its the contracts are captured immutably. Since it is done in a public domain, the transaction is fully ``transparent" to the whole public, in which case the responsibilities of each party, as well as the accountabilities,  are clearly defined (in another word, anyone could ``see" and checked whether Alice or Bob are performing their promised performances). 

Under such conditions, the ``promise" made by Alice to Bob, could now be verified and proofed by any other people at any time on real time basis. Based on that arrangements, Bob, could then ``transfer" his rights (over Alice's promise), to Charlie. Charlie is willing to accept the ``transfer" from Bob, without the need of knowing Alice, since he relied on the public records which had been verified and proofed in the public domain, to determine the value of such rights transferred to him by Bob. Bob then could charge a certain payment, say $y_{2}$ to Charlie for him to transfer those rights. Which means Bob could trade that commodity to Charlie, even without referring to Alice.
\end{quote}

In the above narrative, it could be seen that there are no intermediaries involved; all the Payments-versus-Deliveries are carried out by an automated process without any further human alterations; everything that happened (between Alice and Bob, Bob and Charlie) are all recorded immutably, with full transparency and accountability. There are no extra costs imposed by any third party in the process (i.e. no transaction costs), information is complete (as far as the transaction is concerned), there are no externalities involved (assuming free and open competition), and the market is actually fully linked by the lock-in process as described.\footnote{Note that the example suffers from some limitations due to its simplified nature. For the purpose of the discussion here, the example suffices to make home the point that ``trust-less" system of exchanges are possible if it could be transacted and defined in such logical manner.}


\subsection{Blockchain distributed ledger technologies and cryptoken}

The main thrust of the blockchain technology is commonly termed as Distributed Ledger and Database Technologies. It is best described as an encrypted database of ledger transactions which are distributed over the networks as the records which are commonly maintained by the participating computers on the networks, and updated by the networks continuously from the beginning of its creation up to the current time. The network is a fully decentralised system and the maintenance of such network is done on voluntary basis. It is dubbed as the new way of computing, which overcomes many of the defects of the centralised systems of database managements.  Smart Contracts, on the other hand, is actually a set of predetermined codes, which is performing a Turing complete process, whereby the codes are self-executing all the pre-determined rules and process, which one set in place, is no longer changeable or amendable unilaterally by any parties. A cryptoken in short is a set of algorithms based on the Ethereum codes of smart contracts. A cryptoken is basically backed by the ``economic appropriation rights" from the applications of the intellectual property concerned, or easier still to think of it as a \emph{flow through entity}. Furthermore, with such character, the cryptoken could play the role as instruments in a \textit{trust-less} system of exchanges as described above. 

\subsection{Tokenising intellectual property rights as a commodity}

Let's refer back to what Arrow (1962) suggestion where if we could ``commoditise" Intellectual Property Rights, by assigning the ``economic appropriation rights" to it, which then allows such commodities to be exchanged. This could be accomplished by linking directly the economic benefits from those IPR onto Smart Contracts as defined before, and the Smart Contracts will provide the necessary linkages between the economic benefit claims of the IPR by the owners of such IPR, and any such benefits will ``automatically" flows from the usage of the IPR to the ``contracting" parties involved in such IPR. An example of this would be that the IPR owners are giving out IOU's to the ``funders" who funded the activities related to the IPR. Once ``contracted" the flows will be on an automated basis from the IPR owners/users to the ``funders". 

The first step is to define those economic claims and link them to the Smart Contracts; secondly, the terms of the exchanges between the IPR owner and the ``funders" could be linked, by having the ``funders" to make their payments in some currencies such as cryptocurrencies. Since the IOU's are essentially the promise of future delivery of ``payments" against a current payment of cryptocurrencies, the future payments itself could also be made in some cryptocurrencies. Once the whole links could be established, they could then be entered onto a blockchain systems, whereby the ledgers and database will be updated on real time basis. Under such system, a crypto-token to represent the future claims could be generated. For the purpose here, such crypto-token is named as ``cryptoken". In essence, a cryptoken is the whole stack of Smart Contracts, entered onto the blockchain, which will perform its intended purpose on an automated basis as a full ``trust-less" economics exchange program.

Since such cryptoken is enshrined within the ``lock-up" environment, the issues of Payments-vs-Delivery or Post Trade Settlements are solved. Furthermore, if proper valuations in terms of its current value as well as future values, and the risks involved could be quantified, then such cryptoken could be categorised as a commodity (or more aptly, a digital asset), which then could by itself be exchanged and traded. And since, the core subject matter is related to economic's appropriations derived by the IPR, this type of cryptokens are in fact commoditising the IPR itself, as Arrow (1962) had imagined.

The focus of next few section is to explain how such concept could be implemented within an actual setting.

\section{Tokenising metal oxide based ultra-capacitors technology} % Section 2

The main subject of this paper is how to could create a ``trust-less" economic exchanges in real life environment based on a commodity, which in this case is an Intellectual Property Rights over a certain technology used in the manufacturing of metal oxide based ultra-capacitors. This section proceeds with a brief description of the technology (or IPR), and how such IPR be commoditised.\footnote{The set of metal oxide technologies referred to here is based on patents, intellectual property rights, as well as the various developments by Aero Beidou organisation (please see http://aerobeidou.com); and the details of the various patents are available from http://aerobeidou.com/\#Patents} 


\subsection{Metal Oxide based ultra-capacitors technology}

Capacitors had been in the market for a long time, however, only of late the advancement in the technology of capacitors had taken giants leaps whereby the race for the energy storage market had produced what is called as ultra-capacitors. In the case of the specific metal oxide based ultra-capacitors technology dealt with here, took a long time from its early days of discoveries to reach the market place. It is reminiscent of a case where funding and investments took a long time before any positive returns on investments are realised. What further complicates the matter is the intellectual property involved are very delicate, as it involves multi-facets from the scientific discovery to the industrialisation of the manufacturing process and all the way to the end products for the users. All in all, it took almost 30 years from the early discovery, to where it is today.\footnote{We couldn't say that the delay is solely due to scientific discoveries, but in some ways, it also depends on the maturity and demand of the energy storage markets.}

What's interesting though is the development for the market for energy storage devices, which had taken leap and bound over the last ten years, and with the coming of new innovations such as the electric vehicle and Internet of Things (IoT) devices, the market for energy storage with higher power density as well as high energy density had increased exponentially. The issues facing ultra-capacitors comparing to batteries (as the traditional energy storage devices) is akin to a vase (ultra-capacitors) and a bottle (batteries such as Lithium Ion), where a vase could fill in water and dispose water quickly in large volumes (in a matter of seconds), whilst a bottle could take in small volume of water over time, and dispose water off in the same manner. Potent combinations between the two are what the market really need - energy storage devices that can take in energy rather quickly, and pour them out over time (that is the vase is the intake, and the bottle is the one pouring out).

Such feat, however, could not be accomplished easily with existing ultra-capacitors in the market, and the problem lies with the design of such ultra-capacitors. Most existing ultra-capacitors would have limits in terms of its ``shape" which is bulky and round, and size, which is ``large"; on the other hand, most capacitors could only handle lower voltages of up to 2.75 Volts, and can be designed in a cascade or serial arrangement, which requires a balance circuitry. In the case of the metal-oxide based ultra-capacitor, it could typically be stacked up as a single unit, to match the battery's voltage and no extra balance circuit is required. This is the major advantage of the technology.\footnote{There are few strands of technological advancements in ultra-capacitors, namely the carbon based, the graphene based, as well as metal-oxide based. Each had its own advantages and disadvantages in terms of the performance and economics. For details please refer to studies by Burke (2009).}

The metal-oxide based ultra-capacitors invented and designed under the various Intellectual Property Rights addressed here could overcome those limitations as described before. First, it can be designed to the smallest (as small as 15x17mm, thickness 0.25mm with 3.3V with the capacitance of  5mF),\footnote{Thus far this is the smallest size that can be produced, which is the smallest available in the market currently.} it can easily be stacked to reach combined voltage of up to 100 volts (or more, depending on the needs of the applications), there are no upper limits in terms larger sizes (due to stacking capabilities), it shapes are in forms of boxes (similar to Lithium Ion batteries shapes), it can be installed and operated in extreme conditions and environments and many other abilities that the batteries have. As such, effectively it can come together with the batteries, as added component in almost all environments that the batteries could be applied in.\footnote{Such as Lithium Ion batteries. Lithium Ion batteries are used as reference here since it is widely applied within the market, and it represents superior technology when comes to batteries.}

What we could claim here is that the metal oxide ultra-capacitor technology presented here would be the way to solve the needs of energy storage requirements today and as the technology is being further developed, it could potentially satisfy the needs of the future. As in most technologies in computers and electronics, it could be moving on the path of the famous ``Moore's Law" whereby it could be further decreased in sizes, whilst increased in the capacities.\footnote{The technology developments would involve continuous collaborations with various other developers, research labs, and universities, in search of higher capacitance materials and processes of manufacturing of the ultra-capacitors -to achieve higher output capability and higher operating temperature sustainability. Please see Burke (2009) for a summary of the technological advancements in ultra-capacitors technology and its usage within the industries.}


\subsection{How the deployment of the ultra-capacitor technology could be commoditised?}

One of the most important issues of any IPR is about its claim to the economic appropriations as explained in earlier sections. Economic appropriations are measured by the economic performance, which performs the final validation of such claims. The question now would be - what are the specific claims that the metal-oxide ultra-capacitors under study here could lay its claims on the performance, as in comparisons to others in the same field (i.e. other ultra-capacitors technologies)? The answer lies in the basic feature of standard measurement in any ultra-capacitors, which is farad (symbol: F), the SI derived unit of electrical capacitance, which is the ability of a body to store an electrical charge. Farad is derived from the English physicist Michael Faraday, who first proposed such unit of measurement. 

Farad could be measured in many ways, and one of the most simplest measure is as follows:

\begin{equation} \label{farad}
farad = \frac{Ampere * time\ in\ seconds}{Voltage}
\end{equation}

The capacitance is produced within the ultra-capacitor from the materials used and the methods of its applications within any specific units. In the case under study here, the materials used are a mix of metal oxides, namely Ruthenium, Nickel and Titanium oxides, combined with Tantalum, a Rare Earth element. Under the patents involved, it had been documented that such materials, being applied in a certain manner, combined with the methods of constructing the anodes, cathodes, and electrolytes, would produce a certain farad per gram of ultra-capacitor cells. These cells are the one being produced in the production lines, which are directly observable and measurable.

First, let us define the materials used in the manufacturing as $x$, which is a unit measure of the combined weights, in units of gram:

\begin{equation}  \label{x}
x =  materials\ weights\ in\ gram
\end{equation}

The metal oxides then are converted based on the technology, into the capacitance in the ultra-capacitor, which is measured by the \textit{``Capacitance Conversion Ratio"}, represented by the symbol $\alpha(x)$ below:

\begin{equation}  \label{faradpergram}
\alpha(x) = \frac{farad}{x}
\end{equation}

The manufacturing process converts ``essential raw materials" as its input, $y$, which is measured in its weights in gram, into the ultra-capacitor cells which produce its rated performance. Such inputs of $y$ would then produce the unit farads as its output based on a certain efficiency which is termed as \textit{``Capacitance Efficiency Ratio"}, represented by the symbol $\beta$, which could be described by the following equation:

\begin{equation}
\beta(y) = \frac{y}{cell}
\end{equation}

Now we could write the farad equation for the ultra-capacitor cells to be produced within the manufacturing process as:

\begin{equation} \label{faradprocess}
f(\alpha(x), \beta(y)) = \frac{\alpha(x)}{\beta(y)} = \frac{farad}{cell}
\end{equation}

Essential parts of what's the technology contributes are the \textit{Capacitance Conversion Ratio} and the manufacturing technology provides the \textit{Capacitance Efficiency Ratio}. In order for any of the formulas to have economic meaning, the measures for $\alpha(x)$ and $\beta(y)$ needs to be disclosed by the IPR claims, which means that they are required to be made public. What goes on in terms of the usage of exact raw materials to achieve those $\alpha(x)$ and $\beta(y)$ are not revealed, since it is hidden inside the $f(\alpha(x), \beta(y))$ formula, and effectively, $x$ and $y$ as units in grams canceled each other out (and hence subsumed within the farad per cell unit measures).\footnote{Notice that in the formula, $x$, the exact amount of the ``essential raw materials" conversion rates need not be disclosed publicly, since it subsumed inside the formula. That is neat, since there are publicly disclosed claims and could be verified claim from the data (of performance), and yet the trade secrets could remain hidden.} These measures could be time subscripted, to show improvements over time (i.e. as the technology progressed), and it can also be measured and tracked progressively.

The farad equation links to the cash flows from the production process could be determined by identifying the ratio of revenue from the farads produced over a set period of time. This is done as follows:

\begin{equation} \label{faradrevenue}
F_{t} =  f_{t}* n_{t} * q_{t}
\end{equation}

where $n_{t}$ = total number of cells produced during period $t$ and $q_{t}$ is the revenue per farad to be assigned to each unit of farad produced during period $t$. We could now say that $F_{t}$ represents the total value captured for the total farads produced by the production lines during the period $t$. This total value could be priced in units of financial values, depending on the denominations of $q_{t}$ financial terms, which could be in any currency (such as US Dollar). The present value of all incoming cash flows, $F_{t}$, could be obtained simply by applying the discounted cash flows formula, which is:

\begin{equation} \label{faradpv}
FARAD_{t=0} = \sum_{t=1}^{T}  F_{t}*R_{t}
\end{equation}

where $R_{t}$ is the discount rates vector for each period $t$ used for discounting of the cash flows.

What $FARAD$ equation above does is to map the variables, $f_{t}$, $n_{t}$ and $q_{t}$ to a single estimated value at time $t=0$. The parameters, such as $f_{t}$, $n_{t}$ and $q_{t}$ are disclosed, to allow $FARAD$ to be calculated on real time, once the production schedules are set within the total period T. The production schedules here will be in units of cells to be produced and the measured capacitance per unit of those cells, which are all could be published and announced publicly (and audited physically). The cycle from productions to the revenue flows - if hard coded onto blockchain Smart Contracts on a full automata basis - accomplished what is termed as ``Turing complete" process.\footnote{The notion of Turing completeness is important, even though the claim of 100\% completeness here is still quite premature at the current stage of the technology.} The links from physical manufacturing process to financial measures (which is the price, representing expected present values), ``maps" the underlying process of generations of cash flows onto the what is termed as Arrow-Debreu securities (as described by Arrow and Debreu(1954).\footnote{The notion of direct and exact mapping between Turing complete and Arrow-Debreu complete is a novel idea. We wouldn't address this issue here, as the subject requires much more rigorous treatment than what could be done in this paper.}

What is left now is to encode all inputs onto a ``trust-less" system of Smart Contracts. Financially, what had been described in the above $FARAD$ formula is akin to a sum of the present value of  \textbf{forward sale} of the securities - it is a promise of contributions of future flows of values to $FARAD$ from the IPR owner or users based on specific performance. So what had been accomplished is to \textbf{securitise} the future income flows into forward sale contracts, whereby the deliveries (of the promised performance) delivered over the time period in the set time frames and periods. These forward contracts could then be coded in the \emph{trust-less} systems as described in the previous sections. 



\subsection{How Ethereum blockchain could be used to generate the Smart Contracts for FARAD cryptoken?}

There are few major steps to create the smart contracts and cryptoken for FARAD. First, for the creation of FARAD, ERC20 coding approach of Ethereum is deployed, to create the smart tokens. These smart tokens are then linked to the smart contract, which is flows of revenue from the ultra-capacitor production lines into an escrow box. The smart contract codes are based on the precise formula for farad, which at any time, $t=a$, the values of $FARAD$ could be determined as follows:

\begin{equation}
FARAD_{t=a} = Escrow_{t=a} + \sum_{t=a+1}^{T} F_{t}*R_{t} 
\end{equation}

The valuation could be done by anyone simply by checking the data and developed their own forecast of future rates for the present value calculations (which could be obtained from market wide data, such as a certain premium over prevailing interest rates). The market then decides what is the price of FARAD at time $t=a$, based on views of the future. Those views could be made on their judgment in regards to technological developments, prices of ultra-capacitors in the markets, market competitions, as well as the discount rates that they use to infer their present value calculations.

\section{Securitisation and commoditising of intellectual property rights on Ethereum blockchain via cryptoken }

This section discusses how the flows of Smart Contract using the Ethereum Network blockchain technology be applied to the idea of securitising the economic appropriation rights as described before. 

\subsection{Ethereum blockchain}

Ethereum blockchain is among the leading blockchain platform which promotes the uses of ``Smart Contracts".\footnote{The phrase ``Smart Contracts" was coined by computer scientist Nick Szabo (see Szabo(1997)), with the goal of bringing contract law and related business practices to the design of electronic commerce protocols between strangers on the Internet} The preferred development platform for developing Smart Contract for Ethereum is by using Solidity Platform\footnote{Solidity is a Contract-Oriented Programming Language for Ethereum Platform} on the Ethereum Virtual Machine.\footnote{EVM is a sandboxed runtime environment for Smart Contract on Ethereum Networks}

\subsubsection*{The architecture of $FARAD$ Smart Contracts}

The architecture of the $FARAD$ Smart Contracts is done by separating areas of concerns into their own contract. These contracts are then assembled together to make a solid end contract. All these intermediary contracts are designed to be reusable at a later time, making the development of other smart contracts in the future to be easier and faster by using fully tested contract components.

Smart Contracts that are developed for $FARAD$ program are:

\begin{enumerate}
    \item \textbf{Ownable} - The interface to define ownership of the contract.
    \item \textbf{Owned} - The implementation of \textbf{Ownable} interface to change the ownership of the contract.
    \item \textbf{ERC20} - The Ethereum Request for Comment (ERC) Issue 20 Smart Contract interface definitions to ensure fungibility of all ERC20-compatible tokens on the Ethereum Network.
    \item \textbf{ERC20Token} - The implementation of ERC20.
    \item \textbf{Administerable} - The Contract for administering manufacturing process, and the escrow account management.
    \item \textbf{Dashboard} - Implementing the $FARAD$ manufacturing data and escrow updates.
    \item \textbf{Computable} - The safe implementation of add, subtract and multiply on EVM to prevent overflows.
    \item \textbf{Guarded} - A set of reusable modifiers for other contracts to use.
    \item \textbf{FRDToken} - The $FARAD$ token implementation, including the Dashboard contract.
    \item \textbf{FRDCrowdSale}    - The $FARAD$ crowd sale implementation.            
\end{enumerate}

The above relations are summarised in the diagram attached in the Appendix, showing the relations on each of the contract defined above into a larger FARAD Smart Contract.


\subsubsection*{Key design criteria}

One of the important design criteria is the subject of updating values from the ultra-capacitor manufacturing process, as well as when payment to the Escrow Ledger is to be executed. One thing for sure, this should be a controlled process but should be seen by the public at all times. That is why the design of the \textbf{Dashboard} contract is derived from \textbf{Administrable} contract, \textbf{Guarded} and \textbf{Ownable} contract. The implementations of all crucial functions can be executed by the creator of the contract only. Each time the production volume is generated, the maintainer of $FARAD$ Smart Contract\footnote{Virtue Fintech FZ-LLC is the organisation to maintain $FARAD$ Smart Contract} (the Maintainer), shall call the function that updates the production volume, as well as the balance in the Escrow Ledger account.

All these operations are notified to all event subscribers, and will be notified on the Blockchain through the event notification for each of the action by the Maintainer. There are data inputs as parameters, such as $f_{t}$, $n_{t}$, $q_{t}$, and $T$ as described before. But $x$ and $y$ are not shown in the data push. In fact, what will appear in the data push would only be $t$'s, which is the disclosed Production Schedules Ledger.\\

\subsubsection*{The coding of Ethereum Smart Contracts}

The main base of the coding of Ethereum Smart Contracts for $FARAD$ are as follows:
\begin{enumerate}
    \item Define the parameters of the Smart Contracts.
    \item Define the Data Push variables of the for the Smart Contracts.
    \item Code the Smart Contract - from $FARAD$ to the Production Schedules Ledger of the ultra-capacitors.
    \item Explain the flows, from productions to $FARAD$ in terms of the Escrow Ledger.
    \item Explain how the Values within the Escrow Ledger represents the fundamental valuation of $FARAD$.
\end{enumerate}

As could be seen from the graph, the first creation is by creating a cryptoken called $FARAD$, using ERC20 base codes of Ethereum, which then could be used within a token issuance and sale process. Via the Smart Contract, the cryptoken is linked to the metal oxide intellectual property rights through the production process, which will generate the cash flows based on the production of ultra-capacitor cells within the factory production lines. The records of the manufacturing process will be updated automatically onto the blockchain data, which then triggers payment entry into the Escrow Ledger, which will be made by in a fiat currency (which would then requires push data) or using another token such as Ether, based on a predetermined formula.

\subsection{Novelty of the suggested approach}

The approach suggested should have lots of future development implications for any IPR developers and owners, who would like to get their IPR funded. In particular, many IPR had reached beyond the pilot stage and had begun the industrial applications. Which at such stage, the pull and push, between the push of the market demands for greater disclosures of the IPR and the pull of maintaining the advantage of secrecy within the IPR are the greatest. This so-called tug of war could potentially be reduced by taking the approach of the examples as we have developed for $FARAD$.

There are ways to fill the demand from the market for greater and more transparent disclosures, without the need of opening the whole IPR to deep technical scrutiny. What is important are the claims that have economics appropriations attached to it, must be disclosed and be measurable. Many trade secrets could remain hidden and need not be disclosed, and it can be wrapped within the ``technology" boxes, as long as the mapping from those boxes onto the financial measures are precise and could be determined before hand.

Furthermore, the power of automata within the Smart Contract, will ensure that there is no shirking from the part of IPR owners/users from the promises made; and if they could not perform, whatever claims and the financial results could be audit-able from the blockchain data itself. This element of audit-ability on blockchain is a subject that has lots of future use and benefits. Audit-ability is important when coming to governance and compliance matters. The limitations that the proposed approach thus far is on the automata part of the manufacturing process. For example, at the current stage of development, the need to do a ``data push" at some of the data entry points rather than a pure automation, are required. This would open issues and questions of integrity and audit-ability and some other sorts of questions. This could only be answered by advancing the integration of blockchain programming within the manufacturing process itself - that is by installing automata of data push from the process itself onto the blockchain. This would be among the future stages of development.


\subsection{Future development of $FARAD$ Smart Contracts}

The only thing left now is to close some of the ``open-ended" part of the $FARAD$ Smart Contracts, which among others are as follows:

\begin{enumerate}
    \item To further eliminate any push data onto the blockchain from the production process.
    \item To further eliminate the needs of push data from the production process revenue flows and the Escrow Ledger
    \item To develop governance and audit functions within the blockchain in order to allow anyone (within the blockchain process) to perform any real time audit (market transparency).
    \item To further develop and integrate the value chain and supply chain process into the blockchain ecosystem
    \item To further develop whereby the ``end users" of supply chain and value chain parties, could use the $FARAD$ cryptoken as its currency of transactions related to the products and any related products produced from the $FARAD$ ecosystem. 
\end{enumerate}

Nevertheless we could claim that what we proposed here are novel approach towards linking directly Intellectual Property Rights to it's economic activities, which could then be transferred directly to the end-beneficiaries (i.e. the investors and end-users).

\section{Conclusions}

Investments in technological innovations require platform where the investors and the innovators could trade with each other by exchanging Intellectual Property Rights commodity. For such commodity to have any economic values, it must carry with it the economic appropriation rights claimed by the innovation. The commodity, as presented in this paper, could be structured using the blockchain technology, in forms of crypto-token, or cryptoken. During the first instance, the exchange is between the innovators and initial investors, by having the cryptoken offered to the investors; and thereafter, the initial investors could trade such cryptoken in the market. Later on, if a market could be created, the end products from the innovation itself could be a physical commodity to be traded using the cryptoken as its currency.

An example of how this could be done in practice is the creation of $FARAD$, the first cryptoken or cryptocurrency ever introduced to the market, to have the benefits arising from the technology for manufacturing and production of metal oxide based ultra-capacitors. The economic valuation of $FARAD$ is tied and linked directly to the future cash flows arising from such production process in real time and automatic basis.

In doing so, we answered the problems raised by a great economist, Kenneth Arrow, who brought the notion of such commodity to be created; and for that, we honour another great innovator, Michael Faraday, by naming such commodity in his name, $FARAD$. Both great innovators had their name enshrined in the public domain, and we wish that $FARAD$ 
would be generated in real life, in their honour.
\clearpage

\bibliography{scibib}

\bibliographystyle{Science}

\begin{enumerate}

\item Akerlof, George A., {\it The Market for `Lemons': Quality and Uncertainty and the Market Mechanism.} Quarterly Journal of Economics, The MIT Press, 84(3): pages 488-500. [Akerlof (1970)]

\item Arrow, Kenneth J., {\it Economic Welfare and the Allocation of Resources for Invention}, in the Rate and Direction of Inventive Activity, R. Nelson (ed.), Princeton University Press, 1962. [Arrow(1962)]

\item Arrow, Kenneth J., and Gerard Debreu., {\it Existence of an Equilibrium for a Competitive Economy.} Econometrica, Vol. 22, No. 3, July 1954. [Arrow and Debreu (1954)]

\item Burke, Andrew., {\it Ultracapacitor Technologies and Application in Hybrid and Electric Vehicles.} Institute of Transportation Studies, University of California, Davis, July 2009. [Burke (2009)]

\item Buterin, Vitalik., {\it Ethereum Whitepaper.} Ethereum Foundation, 2014. [Buterin (2014)]

\item Gintis, Herbert., {\it Game Theory Evolving: A Problem-Centered Introduction to Modelling Strategic Interaction.} Princeton University Press, Princeton, New Jersey, 2000. [Gintis (2000)]

\item Grossman, Sanford J, and Joseph E. Stiglitz., {\it On the Impossibility of Informationally Efficient Markets.} The American Economic Review, pages 393-408, June, 1980. [Grossman and Stiglitz (1980)]

\item Harm, Julien, Josh Obregon, Josh Stubbendick., {\it Ethereum vs. Bitcoin.} Creighton University, undated manuscript, retrieved 1 July 2017. [Harm et. all (2017)]

\item Kreps, David M., {\it A Course in Microeconomic Theory.} Princeton University Press, Princeton, New Jersey, 1990. [Kreps (1990)]

\item MacMillan, John., {\it Reinventing the Bazaar: A Natural History of the Markets.} W. W. Norton \& Company, New York, 2002. [MacMillan (2002)]

\item Nakamoto, Satoshi., {\it Bitcoin: Peer-to-Peer Electronic Cash System.} White Paper, 2008. [Nakamoto (2008)]

\item Spence, Michael., {\it Job Market Signalling.} The Quarterly Journal of Economics, Vol. 87, No. 3, pages 355-374, Aug., 1973. [Spence (1973)]

\item Szabo, Nick., {\it Formalizing and Securing Relationship on Public Networks.} Firstmonday.org. First Monday, 1997. [Szabo (1997)] 

\item Vigna, Paul and Michael J. Casey., {\it Cryptocurrency: The Future of Money?} Vintage, London, 2015. [Vigna and Casey (2015)]

\end{enumerate}

\clearpage
\section*{Appendix: FARAD Smart Contract Coding References}

The diagram for FARAD Smart Contract.

\begin{center}
    \includegraphics[scale=0.3]{Cryptoken}
\end{center}
\vspace{5mm}

For the codes, please refer to: https://github.com/VirtueFintech/FaradCryptoken.git

\clearpage



\end{document}




















